% Ovdje mogu stajati komentari (i skoro ništa više).
\documentclass[12pt]{scrartcl}
\usepackage[utf8]{inputenc}
\usepackage[croatian]{babel}
\usepackage{csquotes}
\usepackage{amsmath,amssymb,amsthm}
\MakeOuterQuote{"}
\begin{document}
    %\title \author i \date se moraju definirati, te vrijednosti kasnije koristi \maketitle
    \title{Vježbe iz \LaTeX a}
    \author{Željko Đurić}
    \date{\today}
    \maketitle
    \section{Uvod}
        Ovdje ide stvarni sadržaj dokumenta.
        Ovo su "navodnici".
        Rekao je:
        \enquote {
            Koliko puta sam ti rekao:
                \enquote {
                    Ne lupaj vratima!
                }?
        }.
    \section{Osnovni matematički zapisi}
        Matematika unutar teksta piše se između znakova dolara: $x + 3 = 1 + 2 + x$.
        To je pokrata za \(x + 3 = 1 + 2 + x\). %prazna linija kaze da se treba napraviti razmak.

        Ovo je primjer velike crte --- koja se u hrvatskom jeziku renderira ovako.
        Obična crtica piše se kod sastavljenih riječi: auto-moto-klub.
        Postoji i "međucrtica" koja se uglavnom koristi za raspone: konzultacije 12--13h.

        Trotočka se ne piše ovako: ... Mora biti razmaknutija, ovako: \ldots

        U matematičkom modu stvari su komplicirane, ovisno o tome što nam treba: $x,y,\dotsc,z$.
        Između operatora i relacija $x+y+\dotsb+z$.
        "Multiplikativna" trotočka $xy\dotsm z$.
        Između velikih nanizanih operatora $\sum\sum\dotsi\sum a$ ili $\iint\dotsi\int f$.
        Imamo i $\cdots$, $\vdots$ i $\ddots$ (najčešće u matricama).
\end{document}